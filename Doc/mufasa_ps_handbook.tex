\documentclass[a4paper]{report}
\usepackage{amsmath}
\usepackage{color}

\begin{document}
\title{Mufasa PS Handbook}
\author{Merlijn Wajer \and Raymond van Veneti\"{e}}

\definecolor{typeGreen}{rgb}{0.0, 0.6, 0.0}
\definecolor{typeRed}{rgb}{0.6, 0.0, 0.0}

\maketitle
\tableofcontents

\chapter{Foreword}

HOI DIT IS DE MUFASA FUNCTIE LIJST!

\chapter{Input}

\section{Mouse}

\subsection{Types}

A few variables are exported for working with Mufasa Mouse Functions.

TClickType, which, as the name suggests, defines the click type.
\begin{itemize}
	\item $mouse\_Right = 0$ 
	\item $mouse\_Left = 1$
	\item $mouse\_Middle = 2$
\end{itemize}

TMousePress, which defines if the mouse button is to be down or up.
\begin{itemize}
	\item $mouse\_Up$
	\item $mouse\_Down$
\end{itemize}

%  TClickType = (mouse_Left, mouse_Right, mouse_Middle);
%  TMousePress = (mouse_Down, mouse_Up);                

\subsection{MoveMouse}
\textbf{procedure} {\color{blue}{MoveMouse}}({\color{typeRed}
{in x, y: }}{\color{typeGreen}{Integer}})

MoveMouse moves the mouse pointer to the specified x and y coordinates.

\subsection{GetMousePos}
\textbf{procedure} {\color{blue}{GetMousePos}}({\color{typeRed}
{out x, y: }}{\color{typeGreen}{Integer}})

GetMousePos returns the current position of the mouse in X and Y.

\subsection{HoldMouse}
\textbf{procedure} {\color{blue}{HoldMouse}}({\color{typeRed}
{x, y: }}{\color{typeGreen}{Integer}}; {\color{typeRed}{clickType :}}
{\color{typeGreen}{clickType}})

HoldMouse holds the given mouse button (clickType) down at the specified x,y 
coordinate. If the mouse if not at the given x, y yet, the mouse position
will be set to x, y.

\subsection{ReleaseMouse}
\textbf{procedure} {\color{blue}{ReleaseMouse}}({\color{typeRed}
{x, y: }}{\color{typeGreen}{Integer}}; {\color{typeRed}{clickType :}}
{\color{typeGreen}{clickType}})

ReleaseMouse releases the given mouse button (clickType) at the specified x,y 
coordinate. If the mouse if not at the given x, y yet, the mouse position
will be set to x, y.

\subsection{ClickMouse}
\textbf{procedure} {\color{blue}{ClickMouse}}({\color{typeRed}
{x, y: }}{\color{typeGreen}{Integer}}; {\color{typeRed}{clickType :}}
{\color{typeGreen}{clickType}})

ClickMouse performs a click with the given mouse button (clickType) at the
specified x, y coordinate.

\section{Keyboard}

The Keyboard functions in Mufasa are listed here.
Most of them are quite basic, and can use some improvement.

\subsection{Types}

Most of the low level Keyboard functions use Virtual Keys.

\subsection{Virtual Keys}

List of VK Keys here

\subsection{KeyDown}

\textbf{procedure} {\color{blue}{KeyDown}}({\color{typeRed}
{key: }}{\color{typeGreen}{Word}});

KeyDown sends a request to the Operating System to ``fake'' an event that
causes the Key to be ``down''.
``key'' can be any Virtual Key\footnote{See the section on Virtual Keys}.

\subsubsection{Common pitfalls}

Don't forget that certain keys may require that shift, or another key
is down as well.

\subsection{KeyUp}

KeyUp sends a request to the Operating System to ``fake'' an event that
causes the Key to be ``up''.
``key'' can be any Virtual Key.

\textbf{procedure} {\color{blue}{KeyUp}}({\color{typeRed}
{key: }}{\color{typeGreen}{Word}});

\subsection{PressKey}

\textbf{procedure} {\color{blue}{PressKey}}({\color{typeRed}
{key: }}{\color{typeGreen}{Word}});

PressKey combines KeyDown and KeyUp, to fake a key press.

\subsection{SendKeys}

\textbf{procedure} {\color{blue}{SendKEys}}({\color{typeRed}
{s: }}{\color{typeGreen}{String}});

SendKeys takes a string ``s'', and attempts to send it's complete contents to
the client. It currently only accepts characters ranging from ``A..z''.

\subsection{IsKeyDown}
\textbf{function} {\color{blue}{PressKey}}({\color{typeRed}
{key: }}{\color{typeGreen}{Word}}): {\color{typeGreen}{Boolean}};

IsKeyDown returns true if then give VK key is ``down''.

\subsection{Notes}

There is no IsKeyUp, because this can easily be generated by inverting the
result of IsKeyDown:
\begin{verbatim}
    not IsKeyDown (x)
\end{verbatim}


\chapter{Finding Routines}

\section{Colours}

\subsection{FindColor}

\subsection{FindColorTolerance}

\subsection{FindColorsTolerance}

\section{Bitmaps}

% Dit doe je zelf maar

\section{DTMs}

\subsection{Types}

\subsection{FindDTM}

I CAN HAS DTM FINDING!

\subsection{FindDTMs}

\subsection{FindDTMRotated}

\subsection{FindDTMsRotated}

\subsection{DTMFromString}

\subsection{DTMToString}

\subsection{AddDTM}

\subsection{FreeDTM}

\subsection{GetDTM}

\subsection{tDTMtopDTM}

\subsection{pDTMtopDTM}


\chapter{OCR}

\section{Finding text}

\section{Indentifying text}

\subsection{GetTextAtEx}
DAT IS DIT

\section{Sorting functions}

\section{Math}

\section{Client / Window}

\section{Files}

\section{Web}
\subsection{OpenWebPage}

\chapter{Easter Eggs}
????

Hakuna matata!

Wizzyplugin stuff



\end{document}
